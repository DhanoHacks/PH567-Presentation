\documentclass[
	11pt, % Set the default font size, options include: 8pt, 9pt, 10pt, 11pt, 12pt, 14pt, 17pt, 20pt
	%t, % Uncomment to vertically align all slide content to the top of the slide, rather than the default centered
	aspectratio=169, % Uncomment to set the aspect ratio to a 16:9 ratio which matches the aspect ratio of 1080p and 4K screens and projectors
]{beamer}
\usepackage[style=verbose,backend=biber]{biblatex}
\addbibresource{references.bib}

\graphicspath{{plots/}{images/}} % Specifies where to look for included images (trailing slash required)
\usepackage{booktabs} % Allows the use of \toprule, \midrule and \bottomrule for better rules in tables
\usetheme{Madrid} % A good all-around theme
% \usefonttheme{default} % Typeset using the default sans serif font
% \usepackage{palatino} % Use the Palatino font for serif text
% \usepackage[default]{opensans} % Use the Open Sans font for sans serif text
\useinnertheme{circles}
\title[Universality]{10K Feet View of Universality} % The short title in the optional parameter appears at the bottom of every slide, the full title in the main parameter is only on the title page
\subtitle{PH 567 Presentation}
\author[Aditya \and Dhano \and Sapna]{Aditya Mehta \and Dhananjay Raman \and Sapna} % Presenter name(s), the optional parameter can contain a shortened version to appear on the bottom of every slide, while the main parameter will appear on the title slide
\institute[IITB]{
    \begin{figure}
        \includegraphics[width=0.12\linewidth]{iitblogoclr.png}
    \end{figure}
    Indian Institute of Technology, Bombay
    } % Your institution, the optional parameter can be used for the institution shorthand and will appear on the bottom of every slide after author names, while the required parameter is used on the title slide and can include your email address or additional information on separate lines
\date[Oct 4, 2023]{October 4, 2023} % Presentation date or conference/meeting name, the optional parameter can contain a shortened version to appear on the bottom of every slide, while the required parameter value is output to the title slide
\begin{document}

%----------------------------------------------------------------------------------------
%	TITLE SLIDE
%----------------------------------------------------------------------------------------

\begin{frame}
	\titlepage % Output the title slide, automatically created using the text entered in the PRESENTATION INFORMATION block above
\end{frame}

%----------------------------------------------------------------------------------------
%	TABLE OF CONTENTS SLIDE
%----------------------------------------------------------------------------------------

% The table of contents outputs the sections and subsections that appear in your presentation, specified with the standard \section and \subsection commands. You may either display all sections and subsections on one slide with \tableofcontents, or display each section at a time on subsequent slides with \tableofcontents[pausesections]. The latter is useful if you want to step through each section and mention what you will discuss.

%----------------------------------------------------------------------------------------
%	PRESENTATION BODY SLIDES
%----------------------------------------------------------------------------------------

\section{The Theory} % Sections are added in order to organize your presentation into discrete blocks, all sections and subsections are automatically output to the table of contents as an overview of the talk but NOT output in the presentation as separate slides

%------------------------------------------------

\subsection{What is Universality?}
\frame{\frametitle{Today's Topics}\tableofcontents[subsectionstyle=show/shaded]}

\begin{frame}
	\frametitle{What is Universality?}
	
	\begin{itemize}
        \item Universality is the observation that there are properties for a large class of systems that are independent of the dynamical details of the system \pause
        \item For eg: All gases behave similarly at high temperature and low pressure regardless of the type of molecule
    \end{itemize}

\end{frame}

%------------------------------------------------

\subsection{The Tools}
\frame{\frametitle{Today's Topics}\tableofcontents[subsectionstyle=show/shaded]}

\begin{frame}
	\frametitle{The Tools}
	There are a few properties that we can use to compare and contrast between different (one-dimensional) systems: \pause
    \begin{itemize}
        \item Bifurcation Diagram \pause
        \item Lyapunov Exponents \pause
        \item Feigenbaum Constants
    \end{itemize}

\end{frame}

%------------------------------------------------

\begin{frame}
	\frametitle{Bifurcation Points}
    \begin{itemize}
        \item At a bifurcation point, a small change of the parameter value of the system may cause:
        \begin{itemize}
            \item a large change in the number or stability of the equilibrium points of the system,
            \item the emergence of limit cycles (oscillations),
            \item or chaos to emerge from an attracting orbit
        \end{itemize} \pause
        \item The bifurcation point itself does not represent any dynamical change of the system, but rather a \textcolor{red}{qualitative} change of its behavior as one or more parameters crosses a critical threshold.
    \end{itemize}
\end{frame}

%------------------------------------------------

\begin{frame}
	\frametitle{Bifurcation Diagram}
    \begin{itemize}
        \item A bifurcation diagram shows the values visited or approached asymptotically (fixed points, periodic orbits, or chaotic attractors) of a system as a function of a bifurcation parameter in the system. \pause
        \item For a one dimensional map the bifurcation parameter is shown on the horizontal axis of the plot and the vertical axis shows the set of values of the iterated function visits asymptotically from some initial condition.
    \end{itemize}
\end{frame}

%------------------------------------------------

\begin{frame}
	\frametitle{Example Bifurcation Diagram}
    \begin{figure}
        \includegraphics[width=0.7\linewidth]{logistic_bifurcation_diagram.png}
    \end{figure}
\end{frame}

%------------------------------------------------

\begin{frame}
	\frametitle{Lyapunov Exponents}
    \begin{itemize}
        \item Lyapunov exponents are used to quantify the rate of separation of infinitesimally close trajectories. \pause
        \item Two trajectories in phase space with initial separation vector $\delta_0$ diverge at a rate given by 
        \begin{equation}
            \delta_n = \delta_0 e^{\lambda n}
        \end{equation}
        where $\lambda$ is the Lyapunov exponent and $n$ is the number of iterations. \pause
        \item The Lyapunov exponent is \textcolor{red}{positive} for chaotic systems, \textcolor{blue}{zero} for non-chaotic systems (or bifurcation points), and \textcolor{teal}{negative} for systems that converge to fixed points.
    \end{itemize}
\end{frame}

%------------------------------------------------

\begin{frame}
	\frametitle{Lyapunov Exponents}
    \begin{itemize}
        \item For a one dimensional map $x_{n+1} = f(x_n)$, the Lyapunov exponent is given by
        \begin{equation}
            \lambda = \lim_{n \to \infty} \frac{1}{n} \sum_{i=0}^{n-1} \ln \left| f'(x_i) \right|
        \end{equation}
        where $f$ is the map and $x_i$ is the $i$th iterate of the map. \pause
        \item Typically, the lyapunov exponent is zero at the bifurcation points, and highly negative (Read: $-\infty$) at the superstable points.
    \end{itemize}
\end{frame}

%------------------------------------------------

\begin{frame}
	\frametitle{Example Lyapunov Exponent Plot}
    \begin{figure}
        \includegraphics[width=0.6\linewidth]{logistic_lyapunov_exp.png}
    \end{figure}
\end{frame}

%------------------------------------------------

\subsection{Feigenbaum's Discovery}
\frame{\frametitle{Today's Topics}\tableofcontents[subsectionstyle=show/shaded]}

\begin{frame}
	\frametitle{Feigenbaum's Discovery}
	While developing a quantitative description of period doubling in logistic map, Feigenbaum realised that the precise functional form did not matter. \pause

    \begin{figure}
		\includegraphics[width=0.15\linewidth]{feigenbaum.jpg}
		\caption{Mitchell Feigenbaum, 1987. Photo by Ingbert Grüttner}
	\end{figure}

\end{frame}

%------------------------------------------------

\begin{frame}
    \frametitle{Feigenbaum's Discovery}
    For a large family of Nonlinear Dynamic Systems (which includes the Logistic Map):
    \begin{itemize}
        \item The difference between the bifurcation points decreased at the same geometric rate $\delta$ \pause
        \item Separation between stable daughter cycles decreased at same geometric rate $\alpha$ \pause
        %\item {\color{red} This observation ultimately led to a rigorous proof, using the mathematical methods of the renormalization group borrowed from the theory of critical phenomena, that these geometrical ratios were universal numbers that would apply to the quantitative description of any period-doubling sequence generated by nonlinear maps with a single quadratic extremum.} Not important \pause
        %\item The logistic map and the sine map are just two examples of this large universality class. The great significance of this result is that the global details of the dynamical system do not matter.
        %\item In fact, this universality class extends beyond one-dimensional maps to nonlinear dynamical systems described by more realistic physical models corresponding to two-dimensional maps, systems of ordinary differential equations, and even partial differential equations.
    \end{itemize}
    A thorough understanding of the simple Logistic Map is sufficient for describing both qualitatively, and to a large extent, quantitatively, the period-doubling route to chaos for all systems in the family.

\end{frame}   

%------------------------------------------------

\begin{frame}
	\frametitle{Feigenbaum's Constants Visualized}
    \begin{figure}
        \includegraphics[width=0.6\linewidth]{plots/feigenbaum_constants_example.png}
    \end{figure}
\end{frame}

%------------------------------------------------

\section{Universality in One Dimensional Maps}
\subsection{Overview}
\frame{\frametitle{Today's Topics}\tableofcontents[subsectionstyle=show/shaded]}

\begin{frame}
	\frametitle{Overview}

    We will look at four different one dimensional maps: \pause

    \begin{itemize}
        \item Logistic Map \pause
        \item Sine Map \pause
        \item Quadratic Map \pause
        \item Biquadratic Map
    \end{itemize}

\end{frame}

%------------------------------------------------

\subsection{Logistic Map}
\frame{\frametitle{Today's Topics}\tableofcontents[subsectionstyle=show/shaded]}

\begin{frame}
	\frametitle{Logistic Map: Background}
    
        \begin{itemize}
            \item The logistic map is a polynomial mapping (equivalently, recurrence relation) of degree 2, often cited as an archetypal example of how complex, chaotic behaviour can arise from very simple non-linear dynamical equations. \pause
            \item It was popularized by the biologist Robert May as a discrete-time demographic model. \pause
            \item The logistic map is written as
            \begin{equation}
                x_{n+1} = Ax_n(1-x_n)
            \end{equation}
            where $x_n$ denotes the ratio of existing population to the maximum possible population at discrete time $n$, and $A$ denotes a parameter measuring the rate of growth.
        \end{itemize}

\end{frame}

%------------------------------------------------

\begin{frame}
        \frametitle{Results: Bifurcation Plot}
    \begin{figure}
        \includegraphics[width=0.6\linewidth]{logistic_bifurcation_diagram.png}
    \end{figure}
\end{frame}

%------------------------------------------------

\begin{frame}
        \frametitle{Results: Lyapunov Exponent Plot}
    \begin{figure}
        \includegraphics[width=0.6\linewidth]{logistic_lyapunov_exp.png}
    \end{figure}
\end{frame}

%-------------------------------------------------

\begin{frame}
      \frametitle{Logistic Map: Calculation of Feigenbaum Constant $\delta$}
       Bifurcation points: $x^* = 3, 3.449, 3.544, \dots$ \pause
        \begin{align*}
            \delta_{n} &= \frac{A_{n}-A_{n-1}}{A_{n+1}-A_{n}} \\
            \delta_{2} &= \frac{A_{2}-A_{1}}{A_{3}-A_{2}} = 4.751 \\
            \delta_{3} &= \frac{A_{3}-A_{2}}{A_{4}-A_{3}} = 4.655
        \end{align*}
\end{frame}

%--------------------------------------------------

\subsection{Sine Map}
\frame{\frametitle{Today's Topics}\tableofcontents[subsectionstyle=show/shaded]}

\begin{frame}
        \frametitle{Sine Map: Background}
        \begin{align*}
            x_{n+1} &= f(x_n) \text{ , where $A > 0$, $x \in [0,1]$} \\
            f(x) &= A\text{ }sin(x) \\
            f'(x) &= A\text{ }cos(x)
        \end{align*}
\end{frame}

%---------------------------------------------------

\begin{frame}
        \frametitle{Results: Bifurcation Plot}
    \begin{figure}
        \includegraphics[width=0.6\linewidth]
        {sine_bifurcation_diagram.png}
    \end{figure}
\end{frame}

%-----------------------------------------------------

\begin{frame}
        \frametitle{Results: Lyapunov Exponents Plot}
    \begin{figure}
        \includegraphics[width=0.6\linewidth]
        {sine_lyapunov_exp.png}
    \end{figure}
\end{frame}

%-----------------------------------------------------

\subsection{Quadratic Map}
\frame{\frametitle{Today's Topics}\tableofcontents[subsectionstyle=show/shaded]}

\begin{frame}
        \frametitle{Quadratic Map: Background}
        \begin{align*}
            x_{n+1} &= f(x_n) \text{ , where}\\
            f(x) &= 1 - Ax^2 \\
            f'(x) &= -2Ax
        \end{align*}
\end{frame}

%------------------------------------------------------

\begin{frame}
        \frametitle{Results: Bifurcation Diagram Plot}
        \begin{figure}
            \includegraphics[width=0.6\linewidth]
            {quadratic_bifurcation_diagram.png}
        \end{figure}
\end{frame}

%-----------------------------------------------------

\begin{frame}
        \frametitle{Results: Lyapunov Exponent Plot}
        \begin{figure}
            \includegraphics[width=0.6\linewidth]
            {quadratic_lyapunov_exp.png}
        \end{figure}
\end{frame}

%-------------------------------------------------------

\begin{frame}
        \frametitle{Comparison: Feigenbaum Constant $\delta$}
        \begin{figure}
            \includegraphics[width=0.6\linewidth]
            {feigenbaum_deltas_first_three.png}
        \end{figure}
\end{frame}

%-------------------------------------------------------

\begin{frame}
        \frametitle{Comparison: Feigenbaum Constant $\alpha$}
        \begin{figure}
            \includegraphics[width=0.6\linewidth]
            {feigenbaum_alphas_first_three.png}
        \end{figure}
\end{frame}

%-------------------------------------------------------

\subsection{A Quick Question...}
\frame{\frametitle{Today's Topics}\tableofcontents[subsectionstyle=show/shaded]}

\begin{frame}
        \frametitle{A Quick Question...}
        \begin{center}
            \item \textbf{\large{Is everyone convinced that Feigenbaum constants are universal?}} \pause
            \item \textbf{\large{Let's try another example!}}
        \end{center}
\end{frame}

%-------------------------------------------------------
\subsection{Bi-Quadratic Map}
\frame{\frametitle{Today's Topics}\tableofcontents[subsectionstyle=show/shaded]}

\begin{frame}
        \frametitle{Bi-Quadratic Map: Background}
        \begin{align*}
            x_{n+1} &= f(x_n) \text{ , where}\\
            f(x) &= 1 - Ax^4 \\
            f'(x) &= -4Ax^3
        \end{align*}
\end{frame}

%------------------------------------------------------

\begin{frame}
        \frametitle{Results: Bifurcation Diagram Plot}
        \begin{figure}
            \includegraphics[width=0.6\linewidth]
            {biquadratic_bifurcation_diagram.png}
        \end{figure}
\end{frame}

%-----------------------------------------------------

\begin{frame}
        \frametitle{Results: Lyapunov Exponent Plot}
        \begin{figure}
            \includegraphics[width=0.6\linewidth]
            {biquadratic_lyapunov_exp.png}
        \end{figure}
\end{frame}

%------------------------------------------------------

\begin{frame}
        \frametitle{Comparing $\delta$: Bi-Quadratic Map v/s Quadratic Maps}
        \begin{figure}
            \includegraphics[width=0.6\linewidth]
            {feigenbaum_deltas.png}
        \end{figure}
\end{frame}

%--------------------------------------------------------

\begin{frame}
        \frametitle{Comparing $\alpha$: Bi-Quadratic Map v/s Quadratic Maps}
        \begin{figure}
            \includegraphics[width=0.6\linewidth]
            {feigenbaum_alphas.png}
        \end{figure}
\end{frame}

%-------------------------------------------------------

\begin{frame}
        \frametitle{Ambiguity?}
        \begin{center}
        \item \textbf{\large{Why is this constant different from the Feigenbaum constant?}} \pause
        \item Because the degree of equation is different!
        \item \textbf{\large{Is it still the Feigenbaum constant?}} \pause
        \item Yes, but for a different family of functions!
        \end{center}

\end{frame}

%------------------------------------------------------

\begin{frame}
        \frametitle{General Case}
        \begin{align*}
            x_{n+1} = 1 - A |x_{n}|^{r} \text{, where $\mu > 0$}
        \end{align*}
        Then the Feigenbuam constants for different values of r \footfullcite{briggs:1991} are:
        \begin{figure}
            \includegraphics[width=0.4\linewidth]
            {feigenbaum_for_different_r.png}
        \end{figure}
        \vspace{-5mm}
        \setbeamerfont{footnote}{size=\tiny}
\end{frame}

%--------------------------------------------------------
\section{Summary and Conclusion}
\frame{\frametitle{Today's Topics}\tableofcontents[subsectionstyle=show/shaded]}

\begin{frame}
        \frametitle{Summary and Conclusion}
        \begin{itemize}
        \item Feigenbaum studied one dimensional maps and defined two constants $\delta$ and $\alpha$ related to the period doubling route to chaos
        \item Two systems belonging to the same family share the same Feigenbaum Constants
        \item There are also systems which do not share these constants; they belong to different families
        \end{itemize}
\end{frame}

%--------------------------------------------------------

\begin{frame}
        \frametitle{Summary and Conclusion}
        Period doubling routes to chaos are observed in many real-world systems:
        \begin{itemize}
            \item Convection currents in Liquid Mercury \footfullcite{libchaber:1982}
            \item Chemical Oscillators \footfullcite{doona:1994}
            \item Lasers \footfullcite{tarroja:1991}
        \end{itemize}
        Knowing about the properties which these vastly different systems share brings us one step closer to having a theory of everything :).
\end{frame}

%--------------------------------------------------------

\end{document} 